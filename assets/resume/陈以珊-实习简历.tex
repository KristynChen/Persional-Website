\documentclass[UTF8,letterpaper,11pt]{ctexart}  % only change: ctex + UTF8
% --- same packages controlling margins as English ---
\usepackage{latexsym}
\usepackage[empty]{fullpage}
\usepackage{titlesec}
\usepackage{marvosym}
\usepackage[usenames,dvipsnames]{color}
\usepackage{verbatim}
\usepackage{enumitem}
\usepackage[colorlinks=true,urlcolor=blue,linkcolor=blue]{hyperref}
\usepackage{fancyhdr}
\usepackage{tabularx}

% Better character fitting without visual change
\usepackage{microtype}

% % Slightly tighter line spacing
% \linespread{0.98} % was 1.0

% % Section spacing: small but readable
% \usepackage{titlesec}
% \titlespacing*{\section}{0pt}{8pt}{4pt}  % {left}{before}{after}

% List spacing: moderate, not crammed
\usepackage{enumitem}
\setlist[itemize]{itemsep=2pt, topsep=4pt, parsep=0pt, leftmargin=0.15in}

% ---- margins: identical to English ----
\addtolength{\oddsidemargin}{-0.75in}
\addtolength{\evensidemargin}{-0.75in}
\addtolength{\textwidth}{1.5in}
\addtolength{\topmargin}{-0.7in}
\addtolength{\textheight}{1.4in}

\urlstyle{same}
\raggedbottom
\raggedright
\setlength{\tabcolsep}{0in}

% Sections (same visual style as English)
\titleformat{\section}{\vspace{-4pt}\heiti\raggedright\large}{}{0em}{}[\color{black}\titlerule \vspace{-5pt}]

% ======= your macros (unchanged) =======
\newcommand{\resumeItem}[1]{\item\small{{#1 \vspace{-2pt}}}}
\newcommand{\resumeSubheading}[4]{%
  \vspace{-2pt}\item
  \begin{tabular*}{0.97\textwidth}[t]{l@{\extracolsep{\fill}}r}
    \textbf{#1} & #2 \\
    #3 & #4 \\
  \end{tabular*}\vspace{-7pt}
}
\newcommand{\resumeSubSubheading}[2]{%
  \item
  \begin{tabular*}{0.97\textwidth}{l@{\extracolsep{\fill}}r}
    \textit{\small #1} & \textit{\small #2} \\
  \end{tabular*}\vspace{-7pt}
}
\newcommand{\resumeProjectHeading}[2]{%
  \item
  \begin{tabular*}{0.97\textwidth}{l@{\extracolsep{\fill}}r}
    \small #1 & #2 \\
  \end{tabular*}\vspace{-7pt}
}
\renewcommand\labelitemii{$\vcenter{\hbox{\tiny$\bullet$}}$}
\newcommand{\resumeSubHeadingListStart}{\begin{itemize}[label={}]} % keep same leftmargin via setlist above
\newcommand{\resumeSubHeadingListEnd}{\end{itemize}}
\newcommand{\resumeItemListStart}{\begin{itemize}}
\newcommand{\resumeItemListEnd}{\end{itemize}\vspace{-5pt}}

% ===== Document =====
\begin{document}

\begin{center}
  \textbf{\large \scshape 陈以珊} \\ \vspace{1pt}
  \small (+86) 15000938707 $|$ 微信: kristyn5942 $|$
  \href{mailto:Kristyn921207@gmail.com}{\underline{Kristyn921207@gmail.com}} $|$
  \href{https://kristynchen.github.io/Personal-Website/}{\underline{个人网站}}
\end{center}

% 教育背景
\section{教育背景}
\resumeSubHeadingListStart

  \resumeProjectHeading
    {\textbf{上海交通大学} \quad \quad | \quad 电气与计算机工程学士(ECE) \quad | \quad 中国·上海}
    {2026年8月(预计毕业)}
    \resumeItemListStart
      \resumeItem{\textbf{相关课程}: STAT4060J Computational Methods, STAT4710J Data Science and Analytics using Python}
      \resumeItem{\textbf{奖学金}: 台湾学生奖学金(2023--2025);百贤亚洲未来领袖奖学金(2024--2026)}
    \resumeItemListEnd

  \resumeProjectHeading
    {\textbf{庆应义塾大学} \quad \quad| \quad 计算机科学(交换生) \quad | \quad 日本·东京}
    {2025年3月 -- 2025年8月}

  \resumeProjectHeading
    {\textbf{特鲁瓦科技大学} \quad | \quad 法语(交换生) \quad | \quad 法国·特鲁瓦}
    {2025年1月 -- 2025年2月}

  \resumeProjectHeading
    {\textbf{关西学院大学} \quad \quad | \quad 法律(交换生) \quad | \quad 日本·兵库}
    {2024年1月 -- 2024年2月}

\resumeSubHeadingListEnd


% 项目经历
\section{项目经历}
\small \textit{(以下项目着重于展现数据流处理、模型封装验证以及 Agent 系统架构的能力)}

\resumeSubHeadingListStart
  \resumeProjectHeading
    {\textbf{\href{https://terny.ai/}{terny.ai} —— Agentic AI科研机会匹配系统}}{2025年11月 -- 至今}
    \resumeItemListStart
      \resumeItem{设计并实现基于爬虫与大语言模型的科研信息采集系统,实现对复杂实验室网站、科研动态的自动化解析与内容抽取。}
      %\resumeItem{主导\textbf{Agentic AI 架构}设计与落地,构建多智能体(Multi-Agent)协同流水线,封装意图推断与精准匹配服务,保障底层\textbf{可控AI系统}的稳定性指标。}
      \resumeItem{主导\textbf{Agentic AI 架构}设计与落地,构建多智能体(Multi-Agent)协同流水线,使用多智能体协作完成研究方向理解、学生能力建模与双向匹配决策。}
      \resumeItem{运用 Codex 与 Antigravity 框架整合跨源调用并打通复杂任务编排,系统执行层全面引入事件日志与栈追踪模块。同时使用github进行代码管理与版本控制。}
    \resumeItemListEnd

  \resumeProjectHeading
    {\textbf{基于 GNN 与 Transformer 的材料性质预测引擎}}{2025年12月 -- 至今}
    \resumeItemListStart
      \resumeItem{主导自动化数据处理流,数据涵盖材料学级(OC22/DFT)样本清洗聚合、特征抽离及多维度节点向量化映射计算。}
      \resumeItem{搭建并在代码维度实现基于图神经网络(GNNs)与 Transformer 的计算底座,高度重构并客制化具备跨模型可复用的核心 PyTorch 计算集群代码。}
      %\resumeItem{自研并封装完整的业务指标检测及训练过程管线网络,在确保特征层与张量演变更迭具备内部\textbf{可观测性}基础上,为大规模云测与持续工业部署提供可用技术切口。}
    \resumeItemListEnd

  \resumeProjectHeading
    {\textbf{规模化数据回归引擎计算:结构化房价预测体系}}{2025年9月 -- 2025年12月}
    \resumeItemListStart
      \resumeItem{运用 Numpy 与 Pandas 完成对超 25 万条源数据序列及 60+ 复杂维度特征维度的清洗与聚合抽象工程。}
      \resumeItem{基于 sklearn 实现线性回归模型训练,并且采用 Hinge Gradient Descent 优化,通过超参数调优提升模型表现。}
      \resumeItem{交付具备高复用性的基础推断验证模组,模型最终实现了 0.38 RMSE 的预测水准,项目成绩位列班级第 2。}
    \resumeItemListEnd

  \resumeProjectHeading
    {\textbf{假新闻文本分类网络与态势可视化 Dashboard}}{2025年9月 -- 2025年12月}
    \resumeItemListStart
      \resumeItem{使用Python 清理和处理7万条真假新闻记录,应用文本预处理技术以优化数据质量。}
      \resumeItem{利用TF-IDF 对文本进行特征提取,结合SVM 与梯度下降法建立假新闻检测模型,对数据集进行训练与评估。}
      \resumeItem{通过高级文本预处理和特征选择方法,实现93\% 的准确率,模型表现优于基准模型。 }
      \resumeItem{使用 Shiny 构建交互式 Dashboard,将模型预测结果可视化,实现端侧数据的实时监测与直观查询。}
    \resumeItemListEnd
\resumeSubHeadingListEnd

%\resumeSubHeadingListStart
%  \resumeProjectHeading
%    {\textbf{使用Vibe Coding建立2026年历}}{2025年10月 -- 至今}
%    \resumeItemListStart
%      \resumeItem{使用Vibe coding框架构建年历功能模块,支持用户管理年度目标、月度计划、每日任务和重要提醒。}
%      \resumeItem{为系统设计并实现了直观的用户界面,结合交互式元素,使得日程安排和记录更加便捷与高效。}
%      \resumeItem{目前已完成基础框架的搭建,并实现了月视图、周视图、任务提醒等功能,正在进行用户反馈和优化迭代。}
%    \resumeItemListEnd
%\resumeSubHeadingListEnd

% \resumeSubHeadingListStart
%  \resumeProjectHeading
%    {\textbf{PRP研究项目}}{2023年9月 -- 2024年5月}
%    \resumeItemListStart
%      \resumeItem{在教授指导下,研究霍尔定理和分离定理,进行相关讨论与学习。}
%      \resumeItem{参与每周讨论和反馈会,以提升对复杂主题的理解。}
%    \resumeItemListEnd
%\resumeSubHeadingListEnd


% 领导力与活动
\section{领导力与活动}
\resumeSubHeadingListStart

  \resumeProjectHeading
    {\textbf{青年志愿者新闻宣传部副部长}}{2022年9月 -- 2024年8月}
    \resumeItemListStart
      \resumeItem{参与社区服务与志愿活动,负责慈善与教育类项目的对外宣传,统筹社交媒体内容发布与校园宣传协调工作。}
      \resumeItem{带领30人团队前往云南大理洱源开展为期两周的支教活动,负责前期宣传策划、现场组织协调及成果传播。}
    \resumeItemListEnd

  \resumeProjectHeading
    {\textbf{学生会宣传部荣誉成员}}{2022年9月 -- 2024年8月}
    \resumeItemListStart
      \resumeItem{协助策划并执行全学院范围活动,同时定期设计学院周边商品,包括T恤、帆布袋、徽章等,并负责线上销售与推广。}
    \resumeItemListEnd

    \resumeProjectHeading
    {\textbf{ENRG1000J Introduction to Engineering(工程学导论)助教}}{2023年9月 -- 2025年12月}
    \resumeItemListStart
      \resumeItem{支持 60+ 名学生的工程学导论课程学习与成果展示(Expo),负责带领每周lab,并提供技术建议与项目迭代建议}
      \resumeItem{设计实验室设备快速上手流程,优化新生实践学习路径,使用Python 自动化脚本完成作业批量处理与异常识别,构建高效教学管理体系,连续三次获评「优秀助教」并被邀请留任。}
    \resumeItemListEnd

  %\resumeProjectHeading
    %{\textbf{学生会副主席,E-campus 数字化校园项目负责人}}{2020年12月 -- 2022年6月}
    %\resumeItemListStart
      %\resumeItem{针对学生午餐高峰排队问题提出 E-campus 数字化解决方案,向校长及市政府提案,并通过调研与跨部门协作,于2022年实施线上点餐与自助设备,优化校园运营效率。}
    %\resumeItemListEnd

\resumeSubHeadingListEnd


% 技能
\section{技能}
\begin{itemize}[leftmargin=0.15in,label={}]
  \small \item {
    \textbf{AI应用与工程}: Agentic Workflow, LLM Prompt Engineering, AI-assisted Code Generation(Codex, Antigravity)\\
    \textbf{数据与算法}: Python, pandas, NumPy, SQL, R \\
    \textbf{开发与工具}: Git, Shiny (前端面板展示), MATLAB, C/C++\\
    \textbf{语言}: 中文(母语),英语(流利),日语(会话)
  }
\end{itemize}

\end{document}
